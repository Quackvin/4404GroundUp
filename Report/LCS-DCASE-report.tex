\documentclass[11pt]{article}

% Packages
\usepackage[affil-it]{authblk}		% author affiliations in title
\usepackage[margin=1in]{geometry}	% one inch margins
\usepackage{enumerate}				% numbered list environment
\usepackage{wrapfig}				% text wrapped figures
\usepackage{graphicx}				% figures, better than "graphics" apparently
\usepackage{subcaption}				% captions for sub-figures
\usepackage{amsmath}				% math stuff
\usepackage{fancyhdr}				% custom headers
\usepackage[numbers]{natbib}		% used for citet and other citation formats
\usepackage{booktabs}				% nice table borders
\usepackage{tabularx}				% equal width table columns
\usepackage[hidelinks]{hyperref} 	% puts click-able links in the text, fixes issue with urls that have underscores
\usepackage[defaultlines=2,all]{nowidow}	% prevents orphan and widow lines at start and end of paragraphs
\usepackage{multicol}				% control over using multiple columns
\usepackage{lipsum}					% dummy text

% To-do notes and to-do list
\usepackage[colorinlistoftodos,prependcaption]{todonotes}

% Tables with centred, fixed with columns
\usepackage{array}
\newcolumntype{P}[1]{>{\centering\arraybackslash}p{#1}}		% used in cover page
\newcolumntype{Y}{>{\centering\arraybackslash}X}			% used in tabularx for even column distribution

% Make "References" appear in the table of contents
\usepackage[nottoc]{tocbibind}
\renewcommand{\tocbibname}{References}

% Line spacing
%\usepackage{setspace}
%\onehalfspacing

% Headers and footers
\pagestyle{fancy}
\fancyhf{}
\renewcommand{\headrulewidth}{0pt}
\rhead{CITS4404 Artificial Intelligence \& Adaptive Systems}
\lhead{Team G1}
\setlength{\headheight}{14pt}
\cfoot{\thepage}

% Nice abstract
\renewenvironment{abstract}
{\small
	\begin{center}
		\bfseries \abstractname\vspace{-.5em}\vspace{0pt}
	\end{center}
	\list{}{
		\setlength{\leftmargin}{.5cm}%
		\setlength{\rightmargin}{\leftmargin}%
	}%
	\item\relax}
{\endlist}



\begin{document}
\listoftodos[Things to do]

% Title
\title{
	Applying Learning Classifier Systems to Acoustic Scene Classification: DCASE 2017 Challenge \\
	\vspace{0.1in}
	\large CITS4404 Artificial Intelligence \& Adaptive Systems Team Project
}
\author{Yiyang~Gao~(00000000)}
\author{Aaron~Hurst~(21325887)}
\author{Kevin~Kuek~(00000000)}
\author{Scott~McCormack~(00000000)}
\affil{School of Computer Science and Software Engineering}

\date{3rd November, 2017}

\maketitle

% Abstract
\begin{abstract}
	This will be our abstract \\
	\lipsum*[2]
\end{abstract}

\begin{multicols}{2}

\section{Introduction}


Test citation: \cite{Anderson2016}

motivation for feature choices

- feautres don't vary a lot across a file (low standard deviation)

- need to reduce the number of features



\section{Literature Review}





\subsection{Learning Classifier Systems}





\subsection{DCASE Challenge}





\subsection{Acoustic Scene Classification}





\section{Experiment}



- Feature Extraction

- Description of the code (the one we made outselved and Urbaonwicz's)

- Parameters used







\section{Retults}


- Rate of learning (improvement in accuracy over time)

- Overall results: pairwise, all classes at once (confusion matrices)









\section{Discussion}








\section{Conclusion}









\bibliographystyle{IEEEtranN}
\bibliography{IEEEabrv,References}

\end{multicols}

\end{document}