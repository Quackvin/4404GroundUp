\documentclass[11pt]{article}

% Packages
\usepackage[affil-it]{authblk}		% author affiliations in title
\usepackage[margin=1in]{geometry}	% one inch margins
\usepackage{enumerate}				% numbered list environment
\usepackage{wrapfig}				% text wrapped figures
\usepackage{graphicx}				% figures, better than "graphics" apparently
\usepackage{subcaption}				% captions for sub-figures
\usepackage{amsmath}				% math stuff
\usepackage{fancyhdr}				% custom headers
\usepackage[numbers]{natbib}		% used for citet and other citation formats
\usepackage{booktabs}				% nice table borders
\usepackage{tabularx}				% equal width table columns
\usepackage[hidelinks]{hyperref} 	% puts click-able links in the text, fixes issue with urls that have underscores
\usepackage[defaultlines=2,all]{nowidow}	% prevents orphan and widow lines at start and end of paragraphs
\usepackage{multicol}				% control over using multiple columns
\usepackage{lipsum}					% dummy text

% To-do notes and to-do list
\usepackage[colorinlistoftodos,prependcaption]{todonotes}

% Width between columns
\setlength{\columnsep}{0.8cm}

% Tables with centred, fixed with columns
\usepackage{array}
\newcolumntype{P}[1]{>{\centering\arraybackslash}p{#1}}		% used in cover page
\newcolumntype{Y}{>{\centering\arraybackslash}X}			% used in tabularx for even column distribution

% Make "References" appear in the table of contents
\usepackage[nottoc]{tocbibind}
\renewcommand{\tocbibname}{References}

% Headers and footers
\pagestyle{fancy}
\fancyhf{}
\renewcommand{\headrulewidth}{0pt}
\rhead{CITS4404 Artificial Intelligence \& Adaptive Systems}
\lhead{Team G1}
\setlength{\headheight}{14pt}
\cfoot{\thepage}

% Nice abstract
\renewenvironment{abstract}
{\small
	\begin{center}
		\bfseries \abstractname\vspace{-.5em}\vspace{0pt}
	\end{center}
	\list{}{
		\setlength{\leftmargin}{.5cm}%
		\setlength{\rightmargin}{\leftmargin}%
	}%
	\item\relax}
{\endlist}



\begin{document}

% Title
\title{
	Applying Learning Classifier Systems to Acoustic Scene Classification: DCASE 2017 Challenge \\
	\vspace{0.2in}
	\large CITS4404 Artificial Intelligence \& Adaptive Systems Team Project
}
\author{Yiyang~Gao~(21263128)}
\author{Aaron~Hurst~(21325887)}
\author{Kevin~Kuek~(21307006)}
\author{Scott~McCormack~(21875529)}
\affil{School of Computer Science and Software Engineering}

\date{3rd November, 2017}

\maketitle

% Abstract
\begin{abstract}
	Sound classification is an emerging field of research with multifarious technology and big-data applications. To promote research in this space, the DCASE Challenge was instigated with multiple tasks for competitors to attempt. Here, we have investigated the efficacy of Learning Classifier Systems applied to the problem of Acoustic Scene Classification. This involves predicting the environment in which a sound file was captured. Results indicate that even on a highly abbreviated feature set, Learning Classifier Systems can achieve modest accuracy in comparison to standard solutions.
	\\
\end{abstract}

\begin{multicols}{2}


\section{Introduction}


\todo[inline]{Write introduction}




\section{Background}

This section provides a brief review of learning classifier systems (\ref{sec:LCS}), the DCASE Challenge (\ref{sec:DCASE}) and acoustic scene classification (\ref{sec:ASC}).



\subsection{Learning Classifier Systems}
\label{sec:LCS}

First introduced in the mid-1970s, Learning Classifier Systems (LCSs) are a rule-based machine leaning algorithm with a unique combination of learning mechanisms, including a genetic algorithm (GA) \cite{Butz2015}. The core of an LCS is a population of rules, or \textit{classifiers}, which collectively form the solution to the given problem \cite{Urbanowicz2009}. This population of classifiers is gradually evolved toward an optimal and maximally general set \cite{Urbanowicz2009}.

The motivation for this structure is that, when modelling and attempting to predict the outcome of complex systems, a desirable approach is to develop a distributed population of classifiers -- in the form of rules -- that together form an accurate model \cite[p.~2]{Urbanowicz2009}. Each classifier, then, spans a subspace of the problem, with the population spanning the entire problem. Individual classifiers consist of a condition-action rule which says: 'If a problem instance matches this \textit{condition}, perform this \textit{action}'. Classifier fitness is evaluated based on feedback from the problem (generally referred to as the `environment').

The learning process of a LCS includes a rule discovery method known as covering, a generalisation-pressure effect called subsumption, a GA, fitness-based deletion to maintain a finite-sized population and, in some applications, reinforcement learning \cite{Butz2000}.

The rule discovery method, covering, is used to initialise the population by adding a new classifier whenever a problem instance matches no existing classifier. Subsumption is used to eliminate classifiers with more specific conditions that are no more accurate than others with equivalent, but more general conditions. The GA is used as a secondary rule discovery method that only operates on high performing rules. Deletion is employed to maintain a finite population size by removing poorly performing -- i.e. low fitness -- classifiers. Reinforcement learning may be used as a final step in the learning cycle for problems where feedback from the environment is delayed.

A seminal work in the field of LCSs was the introduction of the eXtended Classifier System \cite{Lanzi2008,Sigaud2007}. This incorporated a number of features which substantially improved the performance of LCSs \cite{Lanzi2008}.

A key distinction amongst LCS applications is between supervised learning, in which feedback from the environment is delayed (such as robot navigation), and offline learning, where feedback is immediate and the correct action known in advance (such as classification tasks). This distinction determines whether reinforcement learning is necessary and affects how classifier accuracy is calculated.



\subsection{DCASE Challenge}
\label{sec:DCASE}

Sound classification, or machine listening, is seen as a promising research field with wide-ranging applications \cite{Mesaros2017}. The DCASE challenge has been established to encourage and support work in this space. The challenge provided participants with standard development (training) and evaluation (testing) datasets \cite{Mesaros2016} and a baseline system for comparison and/or extension \cite{Mesaros2017,DCASE2017asc}.

The Challenge spans four sound recognition tasks: acoustic scene classification, detection of rare sound events, sound event detection in real life audio and large-scale weakly supervised sound event detection \cite{Mesaros2017}. This paper focuses on acoustic scene classification. For this task, the Challenge provides datasets containing sound files obtained across 15 different contexts, such as in a car, library or office \cite{DCASE2017asc}. Each sound file in 3-5 minutes long natively, but was split into multiple 10~second long files for the datasets \cite{DCASE2017asc}. 

A baseline system for this task is provided in Python and uses a neural network with two hidden layers of 50 neurons each to classify sound files \cite{Mesaros2017}. The baseline system also provides support for extracting features from the datasets using what is known as log mel-band energies, as described in Section \todo[inline]{Add reference to Kevin's section} \cite{Mesaros2017}. 





\subsection{Acoustic Scene Classification}
\label{sec:ASC}

The first task of the 2017 DCASE Challenge, and the focus of this paper, is acoustic scene classification. \citeauthor{Barchiesi2015} define this as ``the task of associating a semantic label to an audio stream that identifies the environment in which it has been produced'' \cite[p.~17]{Barchiesi2015}. Potential applications in this domain revolve around context aware smart devices, such as smartphones and hearing aids, that adjust their functioning based on the environments in which they find themselves \cite{Barchiesi2015}. 

The typical approach taken to acoustic scene classification is to segment the original sound file into many, small `frames', calculate a set of features over each frame, use the resulting feature array to train a statistical model and finally apply some decision rule for assigning classification labels \cite[pp.~18-19]{Barchiesi2015}. Many approaches submitted to the 2017 DCASE challenge trained some form of neural network as their `statistical model' \cite{DCASE2017asc}.

The results of the 2017 DCASE Challenge show that the baseline solution provided achieved an accuracy of 74.8\% on the development dataset and 61.0\% on the evaluation set averaged across all sound classifications \cite{DCASE2017asc}. Many entrants successfully outperformed the baseline on both the development and evaluation datasets; however, all algorithms performed worse on the evaluation set compared to the development set \cite{DCASE2017asc}. The best performing solution achieved an accuracy of 87.1\% and 83.3\% on the development and evaluation datasets, respectively \cite{DCASE2017asc,Mun2017}.






\section{Feature Extraction}

[KEVIN'S SECTION]



\section{Experiments}
\label{sec:exp}

Data breakdown

Our approach is good because it is more general (no need to construct statistical models or decision rules) and significantly reduces the number of features that need to be analysed, and therefore the complexity of the algorithm. [THIS IS COMPARING TO STANDARD APPROACH TO ACS]



\subsection{LCS Baseline System}
\label{sec:ryan}

Preliminary investigation into LCSs built by others, led us to sourcing from a sample system hosted on GitHub titled \textit{eLCS}. This Michigan-style classifier was developed as an educational tool which assisted with demonstrating the intricacies and components that contribute towards building a typical LCS. Five demo folders are utilized to show the steps towards building a typical LCS and a dataset consisting of discrete attributes and a starting rule population is provided to run the LCS. The last folder in this system, titled \textit{‘Demo 5’} was utilized as a starting point for our baseline system.





\section{Experiment 2: Adapted XCS(R) Design}
\label{sec:home}

Details of the LCS we made, specific design changes relative to the standard 
implementation for our problem


\subsection{Overview of Approach}




\subsection{Parameters}



\subsection{Problem-Specific Modifications}



{\small \[
\frac{exp_{min}}{3.05}
\left(
\pi/2-\arctan\left(\frac{\textrm{age} - age_{min}}{20} \right) \right)
\]}



%- Feature Extraction, motivation for feature choices (i.e. feautres don't vary a lot across a file (low standard deviation), need to reduce the number of features)
%
%- Description of the code (the one we made outselved and Urbaonwicz's)
%
%- Parameters used
%
%Structure and design of algorithm: generic XCS modified for continuous data and offline learning
%
%Key modifications to basic LCs: (1) added tolerance to subsumption and removed wildcard criteria so that subsumption has some likelihood of occurring, (2) changed deletion criteria so that classifiers with little utility (low match count) can be deleted, motivation was the really low average match count, (3) action set subsumption criteria for most general classifier determined as most wildcards or equal most and largest range, (4) ...
%
%Show graph of function for deletion experience threshold (developed in response to \#2)




\section{Results}


\subsection{Environment Representations}

Alternative feature processing investigated



\subsection{Parameter Tuning}

%- Rate of learning (improvement in accuracy over time)
%
%- Overall results: pairwise, all classes at once (confusion matrices)
%
%
%Theoretical analysis?



Reason why standard deviation and mean are used, what information they carry




\section{Discussion}




Car and office confusion matrix
Plot for std dev and mean (for car and office) to explain
Car, office and city centre (with confusion and plots)
Then add metro station and show results (worse) - explain why it is worse
Show a classifier on a graph to show how it can match instances
Overall accuracy (all 15 features)




\section{Conclusion}

\todo[inline]{Write conclusion}







\bibliographystyle{IEEEtranN}
\bibliography{IEEEabrv,References}

\end{multicols}

\end{document}